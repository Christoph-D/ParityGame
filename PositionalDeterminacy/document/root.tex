\documentclass[11pt,a4paper]{scrartcl}
\usepackage{isabelle,isabellesym}
\usepackage{amsfonts}
\usepackage{xspace}
\usepackage[utf8]{inputenc}
\usepackage[T1]{fontenc}

\typearea{11}
\renewcommand{\bf}{\normalfont\bfseries}
\renewcommand{\rm}{\normalfont\rmfamily}
\renewcommand{\it}{\normalfont\itshape}

\usepackage{pdfsetup}

% urls in roman style, theory text in math-similar italics
\urlstyle{rm}
\isabellestyle{it}

% for uniform font size
%\renewcommand{\isastyle}{\isastyleminor}

\newcommand{\Even}{\textsc{Even}\xspace}
\newcommand{\Odd}{\textsc{Odd}\xspace}

\begin{document}

\title{Positional Determinacy of Parity Games}
\author{Christoph Dittmann\\christoph.dittmann@tu-berlin.de}
\date{\today}
\maketitle

\begin{abstract}
  We formalize a proof of positional determinacy of parity games (a
  two-player graph game) in Isabelle/HOL.  This proof works for both
  finite and infinite games.  We follow the proof in
  \cite{kreutzer2015}, which is based on \cite{zielonka1998}.
\end{abstract}

\tableofcontents
\newpage

\section{Introduction}

Parity games are games played by two players, called \Even and \Odd,
on labelled directed graphs.  Each vertex is labelled with their
player and with a natural number, called its \emph{priority}.

To call this a \emph{parity game}, we only need to assume that the
number of different priorities is finite.  Of course, this condition
is only relevant on infinite graphs.

One reason parity games are important is that determining the winner
is polynomial-time equivalent to the model-checking problem of the
modal $\mu$-calculus, a logic able to express LTL and CTL* properties
(\cite{bradfield2007}).

\subsection{Formal introduction}

Formally, a parity game is $G = (V,E,V_0,\omega)$, where $(V,E)$ is a
directed graph, $V_0 \subseteq V$ is the set of \Even vertices, and
$\omega: V \to \mathbb{N}$ is a function with $|f(V)| < \infty$.

A \emph{play} is a maximal path in $G$.  A finite play is winning for
\Even iff the last vertex is not in $V_0$.  An infinite play is
winning for \Even iff the minimum priority occurring infinitely often
on the path is even.  On an infinite path at least one priority occurs
infinitely often because there is only a finite number of different
priorities.

A vertex $v$ is \emph{winning} for a player~$p$ iff all plays starting
from $v$ are winning for~$p$.  It is well-known that parity games are
\emph{determined}, that is, every vertex is winning for some player.

A more surprising property is that parity games are also
\emph{positionally determined}.  This means that for every vertex $v$
winning for \Even, there is a function $\sigma: V_0 \to V$ such that
all \Even needs to do in order to win from $v$ is to consult this
function whenever it is his turn (similarly if $v$ is winning for
\Odd).  This is also called a \emph{positional strategy} for the
winning player.

We define the \emph{winning region} of player~$p$ as the set of
vertices from which player~$p$ has positional winning strategies.
Positional determinacy then says that the winning regions of \Even and
of \Odd partition the graph.

See \cite{automata2002/kuesters} for a modern survey on positional
determinacy of parity games.  Their proof is based on a proof by
Zielonka \cite{zielonka1998}.

\subsection{Overview}

Here we formalize the proof from \cite{kreutzer2015} in Isabelle/HOL.
This proof is similar to the proof in \cite{automata2002/kuesters},
but we do not explicitly define so-called ``$\sigma$-traps''.  Using
$\sigma$-traps could be worth exploring, because it has the potential
to simplify our formalization.

Our proof has no assumptions except those required by every parity
game.  In particular the parity game
\begin{itemize}
\item may have arbitrary cardinality,
\item may have loops,
\item may have deadends, that is, vertices with no successors.
\end{itemize}

% sane default for proof documents
\parindent 0pt\parskip 0.5ex

% generated text of all theories
\input{session}

\clearpage
\phantomsection
\addcontentsline{toc}{section}{Bibliography}
\bibliographystyle{plain}
\bibliography{root}

\end{document}

%%% Local Variables:
%%% mode: latex
%%% TeX-master: t
%%% End:
